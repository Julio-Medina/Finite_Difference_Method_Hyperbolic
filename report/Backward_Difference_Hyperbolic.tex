\documentclass[a4paper]{article}
\usepackage[spanish,es-tabla]{babel}	% trabajar en español
\spanishsignitems	
%\usepackage{simplemargins}

%\usepackage[square]{natbib}
\usepackage{amsmath}
\usepackage{amsfonts}
\usepackage{amssymb}
\usepackage{bbold}
\usepackage{graphicx}
\usepackage{blindtext}
\usepackage{hyperref}
\usepackage{amsthm}
\newtheorem{theorem}{Teorema}
\newtheorem{lemma}{Lema}
\usepackage{algorithm}
%\usepackage{algorithmic}
\usepackage{algpseudocode}
%\usepackage{algorithm2e}
\usepackage{booktabs}

\setcounter{MaxMatrixCols}{20}

\begin{document}
\pagenumbering{arabic}

\Large
 \begin{center}
\textbf{Método de Diferencias Finitas para ecuaciones hiperbólicas\\}


\hspace{10pt}

% Author names and affiliations
\large
%Lic. Julio A. Medina$^1$ \\
Julio A. Medina\\
\hspace{10pt}
\small  
Universidad de San Carlos\\
Escuela de Ciencias Físicas y Matemáticas\\
Maestría en Física\\
\href{mailto:julioantonio.medina@gmail.com}{julioantonio.medina@gmail.com}\\

\end{center}

\hspace{10pt}

\normalsize
\section{Ecuaciones diferenciales parciales hiperbólicas}
La ecuación diferencial parcial hiperbólica o ecuación de onda viene dada por
\begin{equation}\label{eq::hyperbolic_partial_diff}
\frac{\partial^2 u}{\partial t^2}(x,t) - \alpha^2 \frac{\partial^2 u}{\partial x^2}(x,t)=0,\,\,\,\, 0<x<l, \,\,\,\, t>0.
\end{equation}
sujeta a las condiciones
\begin{equation}
u(0,t)=u(l,t)=0,\,\,\,\text{para} t>0, 
\end{equation}
\begin{equation}
u(x,0)=f(x),\,\,\, \text{y}\,\,\,\frac{\partial u}{\partial t}(x,0)=g(x),\,\,\, \text{para}\,\,\, 0\leq x\leq l,
\end{equation}
donde $\alpha$ es una constante que depende de las condiciones físicas del problema. La ecuación de onda es ubicua en toda la física. \\
\subsection{Método de Diferencias Finitas para la ecuación de onda(hiperbólica)}
De manera similar a los métodos utilizados para resolver las ecuaciones parabólicas y elípticas se selecciona un entero $m>0$ para definir puntos de retículo en el eje x usando $h=l/m$. También se hace una selección para el tamaño del paso temporal $k>0$. Los puntos del retículo $(x_i,t_j)$ están definidos por
\begin{equation}
x=ih\,\,\,\, \text{y} \,\,\, t_j=jk
\end{equation}
para cada $i=0,1,2,\hdots,m$ y $j=0,1,2,\hdots$.\\
Para cada punto interior del retículo, i.e. puntos que no se encuentran en las fronteras del retículo se tiene que la ecuación de onda \ref{eq::hyperbolic_partial_diff} se convierte en 
\begin{equation}\label{eq::wave_equation}
\frac{\partial^2 u}{\partial t^2}(x_i,t_j) - \alpha^2 \frac{\partial^2 u}{\partial x^2}(x_i,t_j)=0
\end{equation}
El método de diferencias finitas se obtiene al encontrar la aproximación para la segunda derivada parcial utilizando el cociente de diferencias centradas(ver \cite{Burden}). Estas aproximaciones vienen dadas por
\begin{equation}\label{eq::difference1}
\frac{\partial ^2 u}{\partial t^2}(x_i,t_j)=\frac{u(x_i,t_{j+1})-2u(x_i,t_j)+u(x_i,t_{j-1})}{k^2}-\frac{k^2}{12}\frac{\partial^4 u}{\partial t ^4}(x_i,\mu_j)
\end{equation}
donde $\mu_j \in (t_{j-1},t_{j+1})$ y
\begin{equation}\label{eq::difference2}
\frac{\partial ^2 u}{\partial x^2}(x_i,t_j)=\frac{u(x_{i+1},t_{j})-2u(x_i,t_j)+u(x_{i-1},t_{j})}{h^2}-\frac{h^2}{12}\frac{\partial^4 u}{\partial x ^4}(\xi_i,t_j)
\end{equation}
donde $\xi_i \in (x_{j-1},x_{j+1})$. Sustituyendo \ref{eq::difference1} y \ref{eq::difference2} en \ref{eq::wave_equation} se obtiene
\begin{equation}
\begin{aligned}
\frac{u(x_i,t_{j+1})-2u(x_i,t_j)+u(x_i,t_{j-1})}{k^2}-\alpha^2 \frac{u(x_{i+1},t_{j})-2u(x_i,t_j)+u(x_{i-1},t_{j})}{h^2}\\
=\frac{1}{12}\Bigg[ k^2 \frac{\partial^4 u}{\partial t ^4}(x_i,\mu_j) -\alpha^2 h^2 \frac{\partial^4 u}{\partial x ^4}(\xi_i,t_j)  \Bigg]
\end{aligned}
\end{equation}
Despreciando el término de error que involucra a la 4ta. derivada en el tiempo y la posición se obtiene la ecuación de diferencias
\begin{equation}\label{eq::difference_equation}
\frac{w_{i,j+1}-2w_{i,j}+w_{i,j-1}}{k^2}-\alpha^2\frac{w_{i+1,j}-2w_{i,j}+w_{i-1,j}}{h^2}=0
\end{equation}
Definiendo $\lambda=\frac{\alpha k}{h}$ se puede reescribir \ref{eq::difference_equation} de la siguiente manera
\begin{equation}\label{eq::difference_equation_2}
w_{i,j+1}-w_{i,j}+w_{i,j-1}-\lambda^2 w_{i+1,j}+2\lambda^2 w_{i,j}-\lambda^2 w_{i-1,j}=0
\end{equation}
resolviendo para $w_{i,j+1}$, la aproximación más adelantada en el tiempo, para obtener
\begin{equation}\label{eq::difference_equation_3}
w_{i,j+1}=2(1-\lambda^2)w_{i,j}+\lambda^2(w_{i+1,j}+w_{i-1,j})-w_{i,j-1}.
\end{equation}
Esta ecuación es valida para cada $i=1,2,\hdots,m-1$ y $j=1,2,\hdots$. Las condiciones de frontera dan
\begin{equation}\label{eq::condition_1}
w_{0,j}=w_{m,j}=0,\,\,\,\text{para cada }j=1,2,3,\hdots,
\end{equation}
y las condiciones iniciales implican que
\begin{equation}
w_{i,0}=f(x), \,\,\, para cada i=1,2,\hdots,m-1
\end{equation}
Esto se puede expresar convenientemente en forma matricial, resultando en 
\begin{equation}\label{eq::matrix_difference_equation}
\begin{bmatrix}
w_{1,j+1}\\
w_{2,j+1}\\
w_{3,j+1}\\
\vdots\\
w_{m-1,j+1}\\
\end{bmatrix}
=
\begin{bmatrix}
2(1-\lambda^2) & \lambda^2 & 0 &\dots&0 \\
\lambda^2     &\ddots    & \ddots& \ddots&\vdots\\
0&\ddots&\ddots&\ddots&0\\
\vdots&\ddots&\ddots&\ddots&\lambda^2 \\
0&\dots&0&\lambda^2 &2(1-\lambda^2)
\end{bmatrix}
\begin{bmatrix}
w_{1,j}\\
w_{2,j}\\
w_{3,j}\\
\vdots\\
w_{m-1,j}\\
\end{bmatrix}-
\begin{bmatrix}
w_{1,j-1}\\
w_{2,j-1}\\
w_{3,j-1}\\
\vdots\\
w_{m-1,j-1}\\
\end{bmatrix}
\end{equation}
Las ecuaciones \ref{eq::difference_equation} y \ref{eq::difference_equation_3} implican que el $(j+1)$ ésimo paso temporal requiere conocimiento de los pasos temporales $j$ y $(j-1)$. Esto produce un problema inicial ya que los valores para $j=0$ están dados por la condición \ref{eq::condition_1}, pero los valores para $j=1$ necesarios para encontrar $w_{i,2}$ deben de ser hallados de la condición inicial para la velocidad, o de la primera derivada temporal
\begin{equation}
\frac{\partial u}{\partial t}(x,0)=g(x),\,\,\,0\leq x \leq l.
\end{equation}
Un acercamiento es reemplazar $\frac{\partial u}{\partial t}$ por una aproximación de una diferencia adelantada,
\begin{equation}
\frac{\partial u}{\partial t}(x_i,0)=\frac{u(x_i,t_1)-u(x_i,0)}{k}-\frac{k}{2}\frac{\partial^2 u}{\partial t^2}(x_i,\tilde{\mu}_i)
\end{equation}
para algún $\tilde{\mu}_i \in (0,t_1)$. Resolviendo para $u(x_i,t_1)$ se obtiene
\begin{equation}
\begin{aligned}
u(x_i,t_1)&=u(x_i,0)+k\frac{\partial u}{\partial t}(x_i,0)+\frac{k^2}{2}\frac{\partial^2 u}{\partial t^2}(x_i,\tilde{\mu}_i)\\
&=u(x_i,0)+kg(x_i)+\frac{k^2}{2}\frac{\partial^2 u}{\partial t^2}(x_i,\tilde{\mu}_i)
\end{aligned}
\end{equation} 
Omitiendo el error de truncación se obtiene la aproximación
\begin{equation}
w_{i,1}=w_{i,0}+kg(x_i),\,\,\,\,\text{para cada} i=1,2,\hdots,m-1.
\end{equation}
Sin embargo está aproximación tiene error de truncación $O(k)$ mientras que el error en la ecuación \ref{eq::difference_equation_3} es del orden $O(k^2)$
\subsection{Mejorando la aproximación inicial}
Para obtener una mejor aproximación para $u(x_i,0)$ se expande $u(x_i,t_1)$ en un segundo polinomio de McLaurin en t, con esto se tiene
\begin{equation}
u(x_i,t_1)=u(x_i,0)+k\frac{\partial u}{\partial t}(x_i,0)+\frac{k^2}{2}\frac{\partial^2 u}{\partial t^2}(x_i,0)+\frac{k^3}{6}\frac{\partial^3 u}{\partial t^3}(x_i,\hat{\mu}_i),
\end{equation}
para algún $\hat{\mu}_i \in (0,t_1)$. Si $f''$ existe entonces 
\begin{equation}
\frac{\partial^2 u}{\partial t^2}(x_i,0)=\alpha^2 \frac{\partial^2 u}{\partial x^2}(x_i,0)=\alpha^2\frac{\partial^2 f}{\partial x^2}(x_i)=\alpha^2 f''(x_i)
\end{equation}
además
\begin{equation}
u(x_i,t_1)=u(x_i,0)+kg(x_i)+\frac{\alpha^2 k^2}{2}f''(x_i)+\frac{k^3}{6}\frac{\partial ^3 u}{\partial ^3}(x_i,\hat{\mu}_i.)
\end{equation}
Esto produce un error del orden $O(k^3)$:
\begin{equation}\label{eq::aprox_1}
w_{i,1}=w_{i,0}+kg(x_i)+\frac{\alpha^2 k^2}{2}f''(x_i).
\end{equation}
Si $f \in C^4[0,1]$ pero no se tiene $f''(x_i)$ disponible, se puede usar la ecuación de diferencias  
\begin{equation}\label{eq::second_derivative_diff}
f''(x_0)=\frac{1}{h^2}[f(x_0-h)-2f(x_0)+f(x_0+h)]-\frac{h^2}{12}f^{(4)}(\xi)
\end{equation}
para algún $\xi$, donde $x_0-h<\xi<x_0+h$. Usando \ref{eq::second_derivative_diff} se puede reescribir \ref{eq::aprox_1} como 
\begin{equation}
f''(x_i)=\frac{f(x_{i+1}-2f(x_i)+f(x_{i-1}))}{h^2}-\frac{h^2}{12}f^{(4)}(\tilde{\xi}_i),
\end{equation}
para algún $\tilde{\xi}_i \in (x_{i-1},x_{i+1})$. Esto implica que 
\begin{equation}
u(x_i,t_1)=u(x_i,0)+kg(x_i)+\frac{k^2 \alpha^2}{2h^2}[f(x_{i+1})-2f(x_i)+f(x_{i-1})]+O(k^3 +h^2k^2)
\end{equation}
Con la definición de $\lambda=k\alpha/h$ se obtiene
\begin{equation}
\begin{aligned}
u(x_i,t_1)&=u(x_i,0)+kg(x_i)+\frac{\lambda^2}{2}[f(x_{i+1})-2f(x_i)+f(x_{i-1})]+O(k^3 +h^2k^2)\\
&=(1-\lambda^2)f(x_i)+\frac{\lambda^2}{2}f(x_{i+1})+\frac{\lambda^2}{2}f(x_{i-1})+kg(x_i)+O(k^3 +h^2k^2).
\end{aligned}
\end{equation}
Con esto se obtiene finalmente la mejora a la aproximación inicial, en la forma de una ecuación de diferencias
\begin{equation}\label{eq::aprox_improvement}
w_{i,1}=(1-\lambda^2)f(x_i)+\frac{\lambda^2}{2}f(x_{i+1})+\frac{\lambda^2}{2}f(x_{i-1})+kg(x_i)
\end{equation}
que puede usarse para hallar $w_{i,1}$, para cada $i=1,2,\hdots, m-1$. Para hallar las aproximaciones subsecuentes se utiliza el sistema \ref{eq::matrix_difference_equation}.

\section{Algoritmo de diferencias atrasadas para la ecuación de onda}
Con el desarrollo de la sección anterior se tienen las herramientas para escribir un algoritmo que resuelva ecuaciones diferenciales parciales hiperbólicas de la forma \ref{eq::hyperbolic_partial_diff}. El esquema a seguir es el siguiente
\begin{itemize}
\item Construir el sistema de ecuaciones inicial \ref{eq::matrix_difference_equation}
\item Utilizar la mejora en la aproximación inicial para $w_{i,1}$, ecuación \ref{eq::aprox_improvement}
\item Realizar multiplicaciones matriciales y adición vectorial para cuantos pasos en el tiempo se requiera.
%\item Iterar en el tiempo cuantos pasos se requieran.
\end{itemize}
\subsection{Algoritmo de diferencias atrasadas para ecuación de calor(parabólica)}
Para resolver el problema definido en \ref{eq::hyperbolic_partial_diff} y usando el esquema definido anteriormente conjuntamente con la ecuacion \ref{eq::aprox_improvement} se puede obtener un algoritmo relativamente sencillo para resolver el problema de ecuación de onda
\begin{algorithm}[H]
\caption{Finite Difference Linear System}\label{alg::fininte_difference_hyperbolic}
\begin{algorithmic}[H]
\Function{finite\_difference\_method\_hyperbolic}{$l,T,\alpha,N,m,f$}
\State $h \gets l/m$
\State $k \gets T/N$
\State $\lambda \gets \alpha k /h$
\State $x\gets \text{numpy.linspace}(0, l, m+1)$
\State $t \gets \text{numpy.linspace}(0, T, N+1)$
\State $A \gets$ zeros $(m-1,m-1)$
\State $w \gets$ zeros $(m-1)$
\State $w_0\gets f(x[-1:1])$
\State $w_1 \gets (1-\lambda^2)f(x[1:-1])+\frac{\lambda^2}{2}(f(x[2:])+f(x[0:-2]))+kg(x[1:-1])$
\For{$i \gets 0$ to $m-2$}
\State $A[i,i]\gets 2(1-\lambda^2)$
\If{$i<m-2$}
\State $A[i,i+1]\gets \lambda^2$
\State $A[i+1,i]\gets\lambda^2$
\EndIf
\EndFor
\For{$j \gets N-2$}
\State $w_j\gets A w_1-w_0$
\State $w_0\gets w_1$
\State $w_1\gets w_j$
\EndFor
\State \textbf{return} $A, w_i,w_0,w,x$
\EndFunction
\end{algorithmic}
\end{algorithm}
La implementación del algoritmo anterior puede encontrarse en GitHub \url{https://github.com/Julio-Medina/Finite_Difference_Method_Hyperbolic/tree/main/code}.
\subsection{Ejemplo}
Usar el método de diferencias atrasadas utilizando el algoritmo \ref{alg::fininte_difference_hyperbolic} con $h=0.1$ y $k=0.05$ para aproximar la solución de la ecuación de calor, usar $h=0.1$ y $k=001$ en el algoritmo
\begin{equation}
\frac{\partial u}{\partial t}(x,t) -\frac{\partial^2 u}{\partial x ^2}(x,t)=0, \,\,\, 0<x<1,\,\,\, t>0,
\end{equation}
sujeta a las condiciones de frontera
\begin{equation*}
u(0,t)=u(1,t)=0,\,\,\, t>0,
\end{equation*}
con condiciones iniciales
\begin{equation}
u(x,0)=\sin{\pi x},\,\,\, 0\leq x \leq 1,\,\,\,\text{y}\,\,\,\frac{\partial u}{\partial t}(x,0)=0,\,\,\, 0\leq x \leq 1,
\end{equation}
y comparar las aproximaciones con el la solución analítica
\begin{equation}
u(x,t)=\sin{(\pi x)}\cos{(2\pi t)}
\end{equation}
Para  resolver este problema se utilizan las funciones creadas el algoritmo \ref{alg::fininte_difference_hyperbolic}, la tabla donde se tabulan los errores se presenta a continuación\\
\begin{table}\label{Tab::table_1}
\begin{tabular}{lrrrr}%
\toprule
i &  $x_i$ &    $w_{i,20}$ &      $u(x_{i,1})$ &  $|u(x{_i,1})-w_{i,20}|$ \\
\midrule
0  &  0.0 &  0.000000 &  0.000000e+00 &       0.000000e+00 \\
1  &  0.1 &  0.309017 &  3.090170e-01 &       5.551115e-17 \\
2  &  0.2 &  0.587785 &  5.877853e-01 &       2.220446e-16 \\
3  &  0.3 &  0.809017 &  8.090170e-01 &       0.000000e+00 \\
4  &  0.4 &  0.951057 &  9.510565e-01 &       3.330669e-16 \\
5  &  0.5 &  1.000000 &  1.000000e+00 &       0.000000e+00 \\
6  &  0.6 &  0.951057 &  9.510565e-01 &       2.220446e-16 \\
7  &  0.7 &  0.809017 &  8.090170e-01 &       0.000000e+00 \\
8  &  0.8 &  0.587785 &  5.877853e-01 &       0.000000e+00 \\
9  &  0.9 &  0.309017 &  3.090170e-01 &       5.551115e-17 \\
10 &  1.0 &  0.000000 &  1.224647e-16 &       1.224647e-16 \\
\bottomrule
\end{tabular}
\end{table}
\section{Conclusiones}
El método de diferencias finitas aplicado a ecuaciones diferenciales parciales hiperbólicas con condiciones de frontera y condiciones iniciales en el tiempo tiene una implementación relativamente sencilla. El método se centra en construir una ecuación matricial para iterar en el tiempo. Con la ayuda teórica de la mejora en la aproximación inicial se consiguen mejores aproximaciones. Como se puede notar en la  tabla \ref{Tab::table_1}, se tiene el comportamiento esperado, es decir que se mantiene la simetría con respecto al punto $t=0.5$ con las aproximaciones con respecto a la solución analítica.


\begin{thebibliography}{99}
%% La bibliografía se ordena en orden alfabético respecto al apellido del 
%% autor o autor principal
%% cada entrada tiene su formatado dependiendo si es libro, artículo,
%% tesis, contenido en la web, etc

\bibitem{Burden} Richard L. Burden, J. Douglas Faires \textit{Numerical Analysis}, (Ninth Edition). Brooks/Cole, Cengage Learning. 978-0-538-73351-9

\bibitem{Medina} Julio Medina. \textit{Método de Diferencias Finitas para ecuaciones elípticas}. \url{https://github.com/Julio-Medina/Finite_Difference_Method}

\bibitem{Varga} Richard S. Varga. \textit{Matrix Iterative Analysis}. Second Edition. Springer. DOI 10.1007/978-3-642-05156-2

%\bibitem{Feynman} 
%\bibitem{Hopfield} J.J. Hopfield. \textit{Neural Networks and physical systems with emergent collective computational abilities}. \url{https://doi.org/10.1073/pnas.79.8.2554}


%\bibitem{McCulloch} Warren S. McChulloch, Walter H. Pitts. \textit{A LOGICAL CALCULUS OF THE IDEAS IMMANENT IN NERVOUS ACTIVITY}. \url{http://www.cse.chalmers.se/~coquand/AUTOMATA/mcp.pdf}



\end{thebibliography}
\end{document}

